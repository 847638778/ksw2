\documentclass{bioinfo}
\copyrightyear{2017}
\pubyear{2017}

\usepackage{hyperref}
\usepackage{amsmath}
\usepackage[ruled,vlined]{algorithm2e}
\newcommand\mycommfont[1]{\footnotesize\rmfamily{\it #1}}
\SetCommentSty{mycommfont}
\SetKwComment{Comment}{$\triangleright$\ }{}

\usepackage{natbib}
\bibliographystyle{apalike}

\begin{document}
\firstpage{1}

\title[Pairwise alignment with DP]{On pairwise alignment with dynamic programming}
\author[Li]{Heng Li}
\address{Broad Institute, 415 Main Street, Cambridge, MA 02142, USA}

\maketitle

\begin{abstract}
This \emph{informal} note introduces important literatures on pairwise
biological sequence alignment with dynamic programming (DP), disambiguates a
few different formulations, investigates the choice of gap cost function and
discusses the implementation of DP-based alignment with a focus on practical
applications. It targets developers who are interested in the history of
DP-based alignment and want to understand it in depth.  Importantly, this note
is not a general tutorial. Audience should be familiar with the basis of
pairwise alignment before reading the note.
\end{abstract}

\begin{methods}
\section{Alignment with dynamic programming}

Let $T$ be the \emph{target} sequence or the \emph{reference} sequence of
length $n=|T|$ and $Q$ be the \emph{query} sequence of length $m=|Q|$. Gaps on
the target sequence are \emph{deletions}; gaps on the query sequence are
\emph{insertions}. Function $s(i,j)$, $0\le i<n$ and $0\le j<m$, gives the
score between the $T[i]$ and $Q[j]$. We will focus on global alignment
algorithms because it is usually more complex than local alignment due to
non-trivial initial conditions.

\subsection{Levenshtein distance}

\emph{Levenshtein distance}~\citep{Levenshtein:1966aa} is the minimum sum of
substitutions, insertions and deletions among all possible alignments between
two sequences. The author proposed the measurement but did not provide an
algorithm to compute the distance. Levenshtein distance is the most common
measure of edit distance. Other variants of edit distances may include fewer
(e.g. no substitutions) or more types of primitive edit operations (e.g.
transpositions).

Levenshtein distance can be computed with dynamic programming (DP). Let
$H_{ij}$ be the distance between prefixes $T[0..i]$ and $Q[0..j]$. The distance
between the full sequences will be $H_{n-1,m-1}$, which can be recursively
computed with
\begin{equation}\label{eq:ed}
H_{ij}=\min\{H_{i-1,j-1}+1-\delta_{T[i],Q[j]}, H_{i-1,j}+1, H_{i,j-1}+1\}
\end{equation}
where $\delta_{ab}$ is the Kronecker delta function, which equals 1 if $a=b$ or
0 otherwise. The initial conditions are: $H_{-1,j}=j+1$ and $H_{i,-1}=i+1$.
This is an $O(mn)$ algorithm.

Practically faster algorithms exist. \citet{Landau:1986aa} found an $O(kn)$
algorithm that guarantees to find the optimal solution if the edit distance is
no larger than $k$. It is particularly fast if the distance is
small~\citep{Sosic:2015aa}.

\citet{Myers:1999aa} made an important observation that if we define
\[\left\{\begin{array}{l}
u_{ij}\triangleq H_{ij}-H_{i-1,j}\\
v_{ij}\triangleq H_{ij}-H_{i,j-1}
\end{array}\right.\]
Eq.~(\ref{eq:ed}) can be transformed to
\begin{equation}
\left\{\begin{array}{l}
z_{ij}=\min\{1-\delta_{T[i],Q[j]},v_{i-1,j}+1,u_{i,j-1}+1\}\\
u_{ij}=z_{ij}-v_{i-1,j}\\
v_{ij}=z_{ij}-u_{i,j-1}
\end{array}\right.
\end{equation}
$u_{ij}$ and $v_{ij}$ only take values $-1$, $0$ or $1$. This enables bit-level
parallelization, resulting in an $O(mn/w)$ algorithm where $w$ is the size of a
machine word in bits. Combined with bounding~\citep{Ukkonen:1985aa}, Myers'
algorithm is significantly faster than Eq.~(\ref{eq:ed}) for long sequences.
Edlib~\citep{Sosic:2017aa} provides an efficient implementation of this
algorithm with added functionality.

\subsection{Linear gap cost}

When \citet{Needleman:1970aa} first proposed to align a pair of biological
sequences with DP, they focused on a linear gap cost function
$\gamma_0(k;e)=k\cdot e$, $e>0$. Let the optimal score up to cell $(i,j)$ be
$H_{ij}$. It can be computed with
\begin{equation}\label{eq:linear}
H_{ij}=\max\{H_{i-1,j-1}+s(i,j), H_{i-1,j}-e, H_{i,j-1}-e\}
\end{equation}
This equation is a generalization of Eq.~(\ref{eq:ed}). It gives matches,
mismatches and gaps different weights.  Typically, we require $2e$ to be larger
than the mismatch cost; otherwise the alignment would always prefer two gaps
over a mismatch -- the final alignment would not contain any mismatches.

It is also possible to apply Myer's bit-parallelis to the linear gap cost
if $s(i,j)$ and $e$ are small integers~\citep{Loving:2014aa}.

\subsection{Classical formulation of affine gap cost}

With a linear cost function, a long gap is considered to arise from a series of
small gaps independently. In evolution, however, a long gap at times results
from one event (e.g. a transposon insertion). A linear gap cost often breaks a
long gap into small pieces and complicates the interpretation of alignment.
Therefore, it is discouraged to use a linear cost to produce alignment for
evolutionarily related sequences.

The limitation of linear cost motivated \citet{Waterman:1976aa} to use a more
general cost function. This work was later folded to \citet{Smith:1981aa},
\emph{the} Smith-Waterman paper. With a general cost, the time complexity to
find the optimal alignment is $O(mn\max\{m,n\})$. \citet{Gotoh:1982aa} showed
that when the cost function takes a form $\gamma_1(k;q,e)=q+k\cdot e$, which is
called \emph{affine gap cost}, it is possible to solve the alignment problem in
$O(mn)$ time. \citet{Altschul:1986aa} fixed an issue in the original Gotoh's
algorithm and introduced the formulation we commonly use today.

\subsubsection{Durbin's formulation}

To give alignment a probablitistic interpretation, \citet{Durbin:1998uq} 
introduced
\begin{equation}\label{eq:durbin}
\left\{\begin{array}{l}
M_{ij}=\max\{M_{i-1,j-1}, E_{i-1,j-1}, F_{i-1,j-1}\} + s(i,j)\\
E_{ij}=\max\{M_{i-1,j}-q, E_{i-1,j}\} - e\\
F_{ij}=\max\{M_{i,j-1}-q, F_{i,j-1}\} - e
\end{array}\right.
\end{equation}
This formulation has a natural connection to pair-HMM with each state having a
clear meaning in alignment. It, however, has one problem: it disallows
transitions between $E$ and $F$ states and thus forbids insertions immediately
followed by deletions (and vice versa). When the gap extension cost $e$ is
smaller than the half of a mismatch cost, transitions between $E$ and $F$ may
yield a better alignment score. It is possible to add transitions between $E$
and $F$ in Eq.~(\ref{eq:durbin}), but in practice, AE86's
formulation~\citep{Altschul:1986aa} will be simpler and faster to implement.

\subsubsection{AE86's formulation}

In Eq.~(\ref{eq:durbin}), if we let:
\[H_{ij}\triangleq\max\{M_{ij},E_{ij},F_{ij}\}\]
Durbin's formulation allowing $E$--$F$ transitions becomes
\begin{equation}\label{eq:ae86-ori}
\left\{\begin{array}{l}
E_{ij}=\max\{H_{i-1,j}-q, E_{i-1,j}\} - e \\
F_{ij}=\max\{H_{i,j-1}-q, F_{i,j-1}\} - e \\
H_{ij}=\max\{H_{i-1,j-1}+S(i,j), E_{ij}, F_{ij}\}
\end{array}\right.
\end{equation}
This is AE86's formulation. In practice, we sometimes compute the cells in the
following order
\begin{equation}\label{eq:ae86}
\left\{\begin{array}{l}
H_{ij}=\max\{H_{i-1,j-1}+S(i,j), E_{ij}, F_{ij}\}\\
E_{i+1,j}=\max\{H_{ij}-q, E_{ij}\} - e \\
F_{i,j+1}=\max\{H_{ij}-q, F_{ij}\} - e
\end{array}\right.
\end{equation}
with initial conditions
\begin{equation}
\left\{\begin{array}{ll}
H_{-1,-1}=0\\
H_{-1,j}=-q-e-j\cdot e & (0\le j<m)\\
H_{i,-1}=-q-e-i\cdot e & (0\le i<n)\\
E_{0j}=-2q-2e-j\cdot e & (0\le j<m)\\
F_{i0}=-2q-2e-i\cdot e & (0\le i<n)
\end{array}\right.
\end{equation}
We don't need $E_{-1,\cdot}$ or $F_{\cdot,-1}$ because Eq.~(\ref{eq:ae86})
does not start with these initial values.

AE86 can be impelmented in different ways depending on the order of computation
and how row scores are stored. Algorithm~\ref{algo:ae86} gives one
implementation. At the beginning of each iteration at line 1, $f=F_{ij}$,
$h=H_{i,j-1}$, $H[j]=H_{i-1,j-1}$ and $E[j]=E_{ij}$. The loop computes
$H$ in cell $(i,j)$, $E$ in the next row and $F$ in the next column.

\begin{algorithm}[tb]
\DontPrintSemicolon
\footnotesize
\KwIn{Targe sequence $T$ and query $Q$; scoring matrix $S(\cdot,\cdot)$ and
affine gap cost $\gamma_1(k;q,e)=q+k\cdot e$}
\KwOut{Best alignment score between $T$ and $Q$}
\BlankLine
\textbf{Function} {\sc AlignScore}$(T,Q,q,e)$
\Begin {
	\For (\Comment*[f]{Generate query profile}) {$a\in\Sigma$} {
		\For{$j\gets0$ \KwTo $|Q|-1$} {
			$P[a][j]\gets S(Q[j],a)$\;
		}
	}
	\For{$j\gets0$ \KwTo $|Q|-1$} {
		$H[j]\gets-q-j\cdot e$\Comment*[r]{$H[j]=H_{-1,j-1}$}
		$E[j]\gets-2q-2e-j\cdot e$\Comment*[r]{$E[j]=E_{0j}$}
	}
	$H[j]\gets 0$\Comment*[r]{$H_{-1,-1}=0$}
	\For{$i\gets0$ \KwTo $|T|-1$} {
		$f\gets-2q-2e-i\cdot e$\Comment*[r]{$f=F_{i0}$}
		$h\gets-q-e-i\cdot e$\Comment*[r]{$h=H_{i,-1}$}
		$p\gets P[T[i]]$\;
		\For{$j\gets0$ \KwTo $|Q|-1$} {
			\nl$s\gets p[j]$\Comment*[r]{$s=S(Q[j],T[i])$}
			$h'\gets\max\{H[j]+s,E[j],f\}$\;
			$H[j]\gets h$\Comment*[r]{$H[j]=H_{i,j-1}$}
			$h\gets h'$\Comment*[r]{$h=H_{ij}$}
			$E[j]\gets\max\{h-q,E[j]\}-e$\Comment*[r]{$E[j]=E_{i+1,j}$}
			$f\gets\max\{h-q,f\}-e$\Comment*[r]{$f=F_{i,j+1}$}
		}
		$H[|Q|]\gets h$\;
	}
	\Return $H[|Q|]$\;
}
\caption{AE86's formulation with affine gap cost}\label{algo:ae86}
\end{algorithm}

\subsubsection{Effect of affine gap cost}\label{sec:affine-eff}

For simplicity, we use one score $a>0$ for all types of matches and one cost
$b>0$ for all types of mismatches. Ignore gaps for now. If the identity between
$T$ and $Q$ is below $1-a/(a+b)$, $T$ and $Q$ will get a negative alignment
score. The $b:a$ ratio sets the minimum identity. Now suppose we have a long
deletion. If $\lceil q/(a+b)\rceil$ or more residues on the query adjacent to
the gap are mismatches but have a perfect match to a subsequence in the gap,
the perfect match will yield a higher alignment score and split the long gap in
two. Gap open cost $q$ determines how easily a long gap to be split into two or
more smaller gaps. Gap extension $e$ is directly related to $E$--$F$
transitions: if $b>2e$, there may be insertions immediately followed by
deletions. $q$ and $e$ together control the total number of gaps. Suppose in a
small region there are $x$ mismatches and one gap. We may achieve a better
score by opening a new gap and add $2y$ gap extensions if $x\cdot b>q+2y\cdot
e$, approximately. We also note that scoring must satisfy $2(q+e)>b$; otherwise
the alignment would not contain any mismatches.

\subsection{Affine gap cost: SIMD acceleration}

SIMD CPU instructions perform one action on a vector of data at the same time.
For example, with SSE2, a type of SIMD, we can compute the sum of two vectors
of sixteen 8-bit integers with one CPU instruction. This is much faster than
summing with sixteen standard instructions. How many data can be processed with
one SIMD instruction depends on the number of bits in the vector and the max
value of each element in the vector. For example, SSE instructions operate on
128-bit vectors. We can process four 32-bit integers or eight 16-bit integers
at the same time. AVX instructions operate on 256-bit vectors. It doubles the
bandwidth of SSE.

SIMD has been used to speed up DP-based pairwise alignment. There are two
general classes of SIMD algorithms: inter-sequence and intra-sequence.
Inter-sequence algorithms~\citep{Rognes:2011aa} align multiple pairs of
sequences at the same time. It is conceptually easier to implement and faster
to run but it is tricky to use with other alignment routines. Intra-sequence
algorithms align one sequence at a time. There are several ways to implement
intra-sequence pairwise alignment, depending on how to organize multiple data
into one vector.  For simplicity, we assume each vector consists of four
elements.

\citet{Wozniak:1997aa} put $(H_{ij},H_{i+1,j-1},H_{i+2,j-2},H_{i+3,j-3})$
into a vector and fills the DP matrix along its diagonal.
\citet{Rognes:2000aa} took a block of row cells into a vector
$(H_{ij},H_{i,j+1},H_{i,j+2},H_{i,j+3})$. \citet{Farrar:2007hs} interleaved rows
into $(H_{ij},H_{i,t+j},H_{i,2t+j},H_{i,3t+j})$ in a striped manner, where
$t=\lfloor(m+3)/4\rfloor$ -- the algorithm collates cells distant apart. In
practice, Farrar's striped algorithm is the fastest and most often
used~\citep{Szalkowski:2008aa,Zhao:2013aa}. \citet{Daily:2016aa} developed a
programming library that implements all three intra-sequence algorithms.

\subsection{Affine gap cost: Suzuki's formulation}

When working with long sequences that yield large alignment scores,
we may need to use 32-bit integers to hold the score arrays. With SSE, we can
only process four cells at a time. Inspired by \citet{Myers:1999aa} and
\citet{Loving:2014aa}, \href{https://github.com/ocxtal}{Hajime Suzuki} proposed
to rewrite Eq.~(\ref{eq:ae86}) with differences between cells:
\begin{equation}
\left\{\begin{array}{l}
u_{ij}\triangleq H_{ij}-H_{i-1,j}\\
v_{ij}\triangleq H_{ij}-H_{i,j-1}\\
x_{ij}\triangleq E_{i+1,j}-H_{ij}\\
y_{ij}\triangleq F_{i,j+1}-H_{ij}
\end{array}\right.
\end{equation}
as
\begin{equation}\label{eq:suzuki}
\left\{\begin{array}{l}
z_{ij}=\max\{s(i,j),x_{i-1,j}+v_{i-1,j},y_{i,j-1}+u_{i,j-1}\}\\
u_{ij}=z_{ij}-v_{i-1,j}\\
v_{ij}=z_{ij}-u_{i,j-1}\\
x_{ij}=\max\{0,x_{i-1,j}+v_{i-1,j}-z_{ij}+q\}-q-e\\
y_{ij}=\max\{0,y_{i,j-1}+u_{i,j-1}-z_{ij}+q\}-q-e
\end{array}\right.
\end{equation}
where $z_{ij}$ is a temporary variable that does not need to be stored.  We can
prove that all variables in these equations are bounded by gap costs and the
extreme match and mismatch scores, but not by the sequence lengths or the peak
alignment score. For small scores, we can encode 16 cells in one SSE vector. In
practice, SSE vectorization of Suzuki's formulation yields better performance
for long sequences.

\subsection{Piece-wise affine gap cost}

Affine gap cost still has issues with long gaps. Recall that an exact match of
length $\lceil q/(a+b)\rceil$ in the middle of a long gap splits the gap into
two if moving this exact match to either edge of the gap leads to mismatches
(Section~\ref{sec:affine-eff}). For noisy reads, it is not infrequent for a
long gap to be split by incidental sequenceing errors. We would prefer to
increase the gap open cost $q$ to avoid such a split, but this would contradict
the high INDEL error rate of some sequencing data.

The root cause of this dilemma is that gaps are caused by two different
mechanisms: evolution which may create a long gap with one event, and
sequencing errors which generate gaps as relatively independent events. A
better gap cost should be concave, such that $\gamma(k+1)-\gamma(k)$ is smaller
with larger $k$.

\citet{Miller:1988aa} found an $O(mn\log\max\{m,n\})$ algorithm for a concave
gap cost function $\gamma(k)$. When $\gamma(k)$ is a piece-wise affine cost
composed of $p$ affine cost functions. their algorithm finds the optimal
alignment in $O(mn\log p)$ time. \citet{Gotoh:1990aa} proposed an $O(mn\cdot
p)$ algorithm, which is simpler and probably faster for small $p$ in practice.

When $p=2$, the two-piece affine cost takes the form
\[
\gamma_2(k;q,e,\tilde{q},\tilde{e})=\min\{q+k\cdot e,\tilde{q}+k\cdot\tilde{e}\}
\]
on the condition that $q+e<\tilde{q}+\tilde{e}$ and $e>\tilde{e}$. Effectively,
this cost function applies $\gamma_1(k;q,e)$ to gaps shorter than
$\lceil(\tilde{q}-q)/(e-\tilde{e})\rceil$ and applies
$\gamma_1(k;\tilde{q},\tilde{e})$ to longer gaps. We can compute the maximal
alignment score under $\gamma_2(k)$ with
\begin{equation}\label{eq:affine2}
\left\{\begin{array}{l}
H_{ij} = \max\{H_{i-1,j-1}+s(i,j),E_{ij},F_{ij},\tilde{E}_{ij},\tilde{F}_{ij}\}\\
E_{i+1,j}= \max\{H_{ij}-q,E_{ij}\}-e\\
F_{i,j+1}= \max\{H_{ij}-q,F_{ij}\}-e\\
\tilde{E}_{i+1,j}= \max\{H_{ij}-\tilde{q},\tilde{E}_{ij}\}-\tilde{e}\\
\tilde{F}_{i,j+1}= \max\{H_{ij}-\tilde{q},\tilde{F}_{ij}\}-\tilde{e}
\end{array}\right.
\end{equation}
Eq.~(\ref{eq:suzuki}) can be extended to work with piece-wise affine cost in a
similar manner.

\end{methods}

\bibliography{aln-dp}
\end{document}
